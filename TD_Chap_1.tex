\documentclass[a4paper,oneside,12pt]{amsbook}
\usepackage[utf8]{inputenc}
\usepackage[english]{babel}
\usepackage{adjustbox}
\usepackage{csquotes}
\usepackage{graphicx}
\usepackage{pgfplots}
\usepackage{textcomp}
\usepackage{url,hyperref,subfiles,amsmath,amsfonts,amssymb,commath,amsthm,parskip,fancyhdr,setspace,tikz-cd, physics,marvosym,tcolorbox}
\usepackage{derivative}
\DeclareMathOperator{\Hom}{Hom}
\DeclareMathOperator{\Ext}{Ext}
\DeclareMathOperator{\Aut}{Aut}
\DeclareMathOperator{\End}{End}
\DeclareMathOperator{\Gal}{Gal}
\DeclareMathOperator{\Gl}{GL}
\DeclareMathOperator{\Sl}{SL}
\DeclareMathOperator{\SO}{SO}
\DeclareMathOperator{\id}{id} 
% \DeclareMathOperator{\tr}{Tr} 
\DeclareMathOperator{\Fr}{Frac}
\DeclareMathOperator{\Ms}{\textit{Meas}}
% \DeclareMathOperator{\ev}{ev}
\DeclareMathOperator{\Cl}{Cl}
\DeclareMathOperator{\Ker}{Ker}
\DeclareMathOperator{\im}{Im}
\DeclareMathOperator{\ur}{ur}
\DeclareMathOperator{\Nm}{N}
% \DeclareMathOperator{\Tr}{Tr}
\DeclareMathOperator{\Frob}{Frob}
\DeclareMathOperator{\ord}{ord} 
% \DeclareMathOperator{\rank}{rank}
\DeclareMathOperator{\supp}{supp}
\DeclareMathOperator{\Rep}{Rep}
% \setlength{\textwidth}{\paperwidth}
% \addtolength{\textwidth}{-2.5in}
% \calclayout
% \renewcommand{\baselinestretch}{1}
\renewcommand\qedsymbol{$\blacksquare$}
\usepackage[OT2, T1]{fontenc}
\newcommand{\N}{\mathbb{N}}
\newcommand{\h}{\mathbb{H}}
\newcommand{\A}{\mathbb{A}}
\newcommand{\I}{\mathbb{I}}
\newcommand{\R}{\mathbb{R}}
\newcommand{\Q}{\mathbb{Q}}
\newcommand{\F}{\mathbb{F}}
\newcommand{\Z}{\mathbb{Z}}
\newcommand{\C}{\mathbb{C}}
\newcommand{\p}{\mathbb{P}}
\newcommand{\Upd}{\operatorname{U}^{\mathrm{M}}}



\newtheorem{thm}{Theorem}[section]
\newtheorem{conj}{Conjecture}[section]
\newtheorem{prop}{Proposition}[section]
\newtheorem{cor}{Corollary}[thm]
\newtheorem{lem}[thm]{Lemma}
\newtheorem{conv}{Convention}[section]
\newtheorem{fact}{Fact}[section]
\theoremstyle{definition}
\newtheorem{defn}{Definition}[section]
\newtheorem{que}{Question}[section]
\newtheorem{exo}{Exercice}[chapter]
\theoremstyle{remark}

\newtheorem{eg}{Example}[section]
\newtheorem*{remark}{Remark}
\newtheorem*{remarks}{Remarks}
\newtheorem{note}{Note}

\setcounter{tocdepth}{1}
 
\title{L1-MI-S1}

\author{TD Chapitre 1}

\newpage









\begin{document}


\maketitle


\setcounter{tocdepth}{1}
\tableofcontents
\mainmatter

\chapter{Ensembles et Applications}
% Vous pouvez trouver les solutions sur cette page. J'ajouterai les solutions des exercices d'hier ce week-end. Mais ne trichez pas, essayez les exercices vous-même, puis nous en discuterons.\\
% Je vous donne un peu de temps pour faire l'exercice, puis je le ferai au tableau. 
\section*{Exercice 1.1}
Nous savons que si \( a \) et \( b \) sont deux nombres réels, et que leur produit \( ab = 0 \), alors soit \( a = 0 \), soit \( b = 0 \). Nous avons le système \((S)\) d'inconnues réelles \(x\) et \(y\) suivant :
\[
(S) :
\begin{cases}
    (x - 1)(y - 2) = 0 & \text{(E1)} \\
    (x - 2)y = 0 & \text{(E2)}
\end{cases}
\]
\section*{Solution}
\subsection*{(1) Détermination des solutions des équations \((E1)\), \((E2)\) et du système \((S)\)}
On dois trouver \( x \) et \( y \) qui satisferont les deux équations simultanément.
\subsubsection*{Solution de l'équation \((E1) : (x - 1)(y - 2) = 0\)}

L'équation \((E1)\) est un produit de deux termes qui est égal à zéro. Nous avons deux possibilités :
\[
(x - 1)(y - 2) = 0 \implies (x - 1) = 0 \quad \text{ou} \quad (y - 2) = 0.
\]
Cela signifie que soit \(x = 1\), soit \(y = 2\). Les solutions sont donc les suivantes :
\[
\begin{cases}
    x = 1 \quad &\text{pour tout } y \in \mathbb{R}, \\
    y = 2 \quad &\text{pour tout } x \in \mathbb{R}.
\end{cases}
\]
Ainsi, l'ensemble des solutions de \((E1)\) est :
\[
E_1 = \{ (1, y) \mid y \in \mathbb{R} \} \cup \{ (x, 2) \mid x \in \mathbb{R} \}.
\]

\subsubsection*{Solution de l'équation \((E2) : (x - 2)y = 0\)}

De la même manière, \((E2)\) est aussi un produit de deux termes égal à zéro. Nous avons :
\[
(x - 2)y = 0 \implies (x - 2) = 0 \quad \text{ou} \quad y = 0.
\]
Cela signifie que soit \(x = 2\), soit \(y = 0\). Les solutions sont donc les suivantes :
\[
\begin{cases}
    x = 2 \quad &\text{pour tout } y \in \mathbb{R}, \\
    y = 0 \quad &\text{pour tout } x \in \mathbb{R}.
\end{cases}
\]
Ainsi, l'ensemble des solutions de \((E2)\) est :
\[
E_2 = \{ (2, y) \mid y \in \mathbb{R} \} \cup \{ (x, 0) \mid x \in \mathbb{R} \}.
\]

\subsubsection*{Solution du système \((S)\)}

Pour résoudre le système \((S)\), nous devons trouver l'intersubsection des ensembles de solutions \(E_1\) et \(E_2\), c'est-à-dire les points qui satisfont à la fois \((E1)\) et \((E2)\). Regardons les solutions possibles :
- \(E_1\) contient les points \(x = 1\) (pour tout \(y\)) et \(y = 2\) (pour tout \(x\)).
- \(E_2\) contient les points \(x = 2\) (pour tout \(y\)) et \(y = 0\) (pour tout \(x\)).

L'intersubsection des deux ensembles donne les points suivants :
\[
S = \{ (1, 0), (2, 2) \}.
\]
Donc, la solution du système \((S)\) est :
\[
S = \{ (1, 0), (2, 2) \}.
\]

\subsection*{(2) Représentation graphique}
Nous allons maintenant représenter graphiquement les solutions obtenues.
\begin{center}
\begin{tikzpicture}[scale=1.0]
    % Axes
    \draw[->] (-1, 0) -- (4, 0) node[right] {\(x\)};
    \draw[->] (0, -1) -- (0, 4) node[above] {\(y\)};
    
    % Grid
    \draw[very thin, gray] (-1, -1) grid (4, 4);
    
    % E1: x = 1
    \draw[blue, thick] (1, -1) -- (1, 4) node[above right] {\(x = 1\)};
    
    % E1: y = 2
    \draw[blue, thick] (-1, 2) -- (4, 2) node[right] {\(y = 2\)};
    
    % E2: x = 2
    \draw[red, thick] (2, -1) -- (2, 4) node[above right] {\(x = 2\)};
    
    % E2: y = 0
    \draw[red, thick] (-1, 0) -- (4, 0) node[right] {\(y = 0\)};
    
    % Intersubsection points
    \fill[black] (1, 0) circle (2pt) node[below left] {$(1, 0)$};
    \fill[black] (2, 2) circle (2pt) node[above left] {$(2, 2)$};
\end{tikzpicture}
\end{center}

- L'équation \((E1)\) représente deux lignes :
  - \(x = 1\), qui est une ligne verticale passant par \(x = 1\),
  - \(y = 2\), qui est une ligne horizontale passant par \(y = 2\).
  
- L'équation \((E2)\) représente également deux lignes :
  - \(x = 2\), une ligne verticale passant par \(x = 2\),
  - \(y = 0\), une ligne horizontale passant par \(y = 0\).

Les points où ces lignes se croisent sont les solutions du système \((S)\), c'est-à-dire les points \((1, 0)\) et \((2, 2)\).

 \section*{Exercice 1.2} Soient deux fonctions \( f : \R \to \R \) et \( g: \R \to \R \), définies par :
\[
f(x) = 3x + 1 \quad \text{et} \quad g(x) = x^2 - 1
\]
On vas calculer \( f \circ g \) et \( g \circ f \), puis vérifier si \( f \circ g = g \circ f \).

\section*{Solution}

\subsection*{1. Calcul de \( f \circ g \)}

La composition \( f \circ g \) signifie que nous devons appliquer \( g \) d'abord, puis appliquer \( f \) au résultat de \( g \). Mathématiquement, cela s'écrit :

\[
(f \circ g)(x) = f(g(x))
\]

Calculons cela étape par étape :

\[
g(x) = x^2 - 1
\]
Appliquons maintenant \( f \) à \( g(x) \), c'est-à-dire \( f(x^2 - 1) \) :

\[
f(x^2 - 1) = 3(x^2 - 1) + 1
\]

Développons cette expression :

\[
f(x^2 - 1) = 3x^2 - 3 + 1 = 3x^2 - 2
\]

Ainsi, nous avons :

\[
(f \circ g)(x) = 3x^2 - 2
\]

\subsection*{2. Calcul de \( g \circ f \)}

De la même manière, la composition \( g \circ f \) signifie que on dois appliquer \( f \) d'abord, puis appliquer \( g \) au résultat de \( f \). Mathématiquement, cela s'écrit :

\[
(g \circ f)(x) = g(f(x))
\]

Calculons cela étape par étape :

\[
f(x) = 3x + 1
\]
Appliquons maintenant \( g \) à \( f(x) \), c'est-à-dire \( g(3x + 1) \) :

\[
g(3x + 1) = (3x + 1)^2 - 1
\]

Développons cette expression :

\[
g(3x + 1) = (9x^2 + 6x + 1) - 1 = 9x^2 + 6x
\]

Ainsi, nous avons :

\[
(g \circ f)(x) = 9x^2 + 6x
\]

\subsection*{3. Comparaison de \( f \circ g \) et \( g \circ f \)}

Nous avons trouvé :
\[
(f \circ g)(x) = 3x^2 - 2
\]
et
\[
(g \circ f)(x) = 9x^2 + 6x
\]


Clairement ils sont deux expression different. Donc, \( f \circ g \neq g \circ f \).

\section*{Résumé}

Nous avons montré que la composition des fonctions \( f \circ g \) et \( g \circ f \) donne des résultats différents. Cela signifie que, en général, la composition de fonctions n'est pas commutative.



\section*{Exercice 1.3}

Soit \( f \) l'application de \( \mathbb{R} \) dans \( \mathbb{R} \) telle que :
\[
f(x) = x^2 + 1
\]
En partant du graphe de \( x \to x^2 \), tracer le graphe de \( f \) et déterminer les ensembles suivants :

\[
f([-3, -1]), \quad f([-3, -1] \cup [-2, 1]), \quad f([-3, -1] \cap [-2, 1]),
\]
\[
f^{-1}((-\infty, -1]), \quad f^{-1}([1, \infty)), \quad f^{-1}((-\infty, 2] \cap [1, \infty)).
\]

\section*{Solution}

\subsection*{1. Tracer le graphe de \( f(x) = x^2 + 1 \)}

Le graphe de \( f(x) = x^2 + 1 \) est une parabole qui est une translation de la parabole \( x^2 \) vers le haut de 1 unité. Cela signifie que le sommet de la parabole est en \( (0, 1) \), et la parabole est symétrique par rapport à l'axe des ordonnées.

\begin{center}
\begin{tikzpicture}
\begin{axis}[
    axis lines = middle,
    xlabel = $x$,
    ylabel = {$f(x) = x^2 + 1$},
    domain=-4:4,
    samples=100,
    ymin=-1, ymax=10,
]
\addplot[color=blue, thick]{x^2 + 1};
\end{axis}
\end{tikzpicture}
\end{center}

\subsection*{2. Calcul des ensembles}

\subsubsection*{a) \( f([-3, -1]) \)}

Nous allons d'abord calculer l'image de l'intervalle \( [-3, -1] \) par la fonction \( f(x) = x^2 + 1 \).

Pour \( x = -3 \), nous avons :
\[
f(-3) = (-3)^2 + 1 = 9 + 1 = 10
\]

Pour \( x = -1 \), nous avons :
\[
f(-1) = (-1)^2 + 1 = 1 + 1 = 2
\]

Ainsi, \( f([-3, -1]) \) est l'intervalle :
\[
f([-3, -1]) = [2, 10]
\]

\subsubsection*{b) \( f([-3, -1] \cup [-2, 1]) \)}

Il suffit de prendre l'image de l'union des intervalles \( [-3, -1] \cup [-2, 1] \).

Pour \( x = -2 \), nous avons :
\[
f(-2) = (-2)^2 + 1 = 4 + 1 = 5
\]

Pour \( x = 1 \), nous avons :
\[
f(1) = 1^2 + 1 = 1 + 1 = 2
\]

Donc, \( f([-3, -1] \cup [-2, 1]) \) est l'intervalle :
\[
f([-3, -1] \cup [-2, 1]) = [1, 10]
\]

\subsubsection*{c) \( f([-3, -1] \cap [-2, 1]) \)}

L'intersection de \( [-3, -1] \) et \( [-2, 1] \) est l'intervalle \( [-2, -1] \).

Pour \( x = -2 \), nous avons :
\[
f(-2) = 5
\]

Pour \( x = -1 \), nous avons :
\[
f(-1) = 2
\]

Ainsi, \( f([-3, -1] \cap [-2, 1]) \) est l'intervalle :
\[
f([-3, -1] \cap [-2, -1]) = [2, 5]
\]

\subsubsection*{d) \( f^{-1}((-\infty, -1]) \)}

La fonction \( f(x) = x^2 + 1 \) prend des valeurs toujours supérieures ou égales à 1. Donc, il n'existe pas de \( x \) tel que \( f(x) \leq -1 \). Cela signifie que :
\[
f^{-1}((-\infty, -1]) = \emptyset
\]

\subsubsection*{e) \( f^{-1}([1, \infty)) \)}

Nous cherchons les \( x \) pour lesquels \( f(x) \geq 1 \). Comme \( f(x) = x^2 + 1 \), et que \( x^2 \geq 0 \) pour tout \( x \), nous avons :
\[
f(x) \geq 1 \quad \text{pour tout} \quad x \in \mathbb{R}
\]
Donc :
\[
f^{-1}([1, \infty)) = \mathbb{R}
\]

\subsubsection*{f) \( f^{-1}((-\infty, 2] \cap [1, \infty)) \)}

L'ensemble \( (-\infty, 2] \cap [1, \infty) = [1, 2] \).

Nous cherchons maintenant les \( x \) tels que \( f(x) \in [1, 2] \). Cela correspond à résoudre l'inéquation :
\[
1 \leq x^2 + 1 \leq 2
\]
Cela équivaut à :
\[
0 \leq x^2 \leq 1
\]
Donc :
\[
-1 \leq x \leq 1
\]

Ainsi, nous avons :
\[
f^{-1}((-\infty, 2] \cap [1, \infty)) = [-1, 1]
\]

\section*{Résumé}

Nous avons calculé et interprété les différents ensembles image et antécédent de la fonction \( f(x) = x^2 + 1 \). Le graphe de cette fonction, une parabole, permet de visualiser ces résultats.

\section*{Exercice 1.5}

Soit $f : \mathbb{R} \to \mathbb{R}$ la fonction définie par $f(x) = x^3 - x$. 
\begin{itemize}
    \item $f$ est-elle injective ?
    \item $f$ est-elle surjective ?
    \item Déterminer $f^{-1}([-1,1])$.
    \item Déterminer $f((0, +\infty))$.
\end{itemize}

\section*{Solution}

\subsection*{Définition}

Soit $f : A \to B$ une fonction définie entre deux ensembles $A$ et $B$. On dit que $f$ est \textbf{injective} si et seulement si :

\[
\forall x_1, x_2 \in A, \quad f(x_1) = f(x_2) \Rightarrow x_1 = x_2.
\]

Autrement dit, une fonction est injective si deux éléments distincts de l'ensemble de départ $A$ ont des images distinctes dans l'ensemble d'arrivée $B$. En termes plus simples, cela signifie qu'aucun élément de $B$ n'est l'image de plus d'un élément de $A$.

\section*{Exemple}

Si $f(x) = 2x + 1$ est une fonction de $\mathbb{R}$ vers $\mathbb{R}$, elle est injective car, pour tout $x_1, x_2 \in \mathbb{R}$, si $f(x_1) = f(x_2)$, alors $x_1 = x_2$.

\subsection*{1. Injectivité de $f$}

La fonction $f$ est définie par $f(x) = x^3 - x$. Pour vérifier si $f$ est injective, nous devons vérifier si $f(x_1) = f(x_2)$ implique $x_1 = $ $x_2$ pour tous $x_1, x_2 \in \mathbb{R}$. 

Par un raisonnement par contraposition, il est équivalent de montrer que si $x_1 \neq x_2$, alors $f(x_1) \neq f(x_2)$. Nous pouvons factoriser $f(x) = x^3 - x$ de la manière suivante :

\[
x^3 - x = x(x^2 - 1) = x(x - 1)(x + 1).
\]

Ainsi, il existe trois solutions à l'équation $x^3 - x = 0$, qui sont $x = -1$, $x = 0$, et $x = 1$. Cela signifie qu'il existe plusieurs valeurs de $x$ qui ont la même image par $f(x)$. Par conséquent, $f(x)$ n'est pas injective.
\subsection*{2. Surjectivité de $f$}

Pour vérifier si $f$ est surjective, nous devons voir si, pour tout $y \in \mathbb{R}$, il existe un $x \in \mathbb{R}$ tel que $f(x) = y$.

Considérons la limite de $f$ lorsque $x \to +\infty$ et $x \to -\infty$ :
\[
\lim_{x \to +\infty} f(x) = \lim_{x \to +\infty} (x^3 - x) = +\infty
\]
\[
\lim_{x \to -\infty} f(x) = \lim_{x \to -\infty} (x^3 - x) = -\infty
\]
Comme $f$ est continue et que ses limites à l'infini sont $+\infty$ et $-\infty$, la fonction $f$ est surjective sur $\mathbb{R}$. Ainsi, $f$ est bien surjective.



\subsection*{4. Détermination de $f((0, +\infty))$}

Soit la fonction $f : \R \to \R$ définie par $f(x) = x^3 - x$. Calculons l'image de l'intervalle $(0, \infty)$ par $f$.

Commençons par étudier la dérivée de $f$. Nous avons :
\[
f'(x) = 3x^2 - 1.
\]
Cherchons les points critiques en résolvant $f'(x) = 0$ :
\[
3x^2 - 1 = 0 \quad \Rightarrow \quad x^2 = \frac{1}{3} \quad \Rightarrow \quad x = \pm \frac{1}{\sqrt{3}}.
\]
Nous nous intéressons à $x > 0$, donc nous prenons $x = \frac{1}{\sqrt{3}}$.

Étudions maintenant le signe de la dérivée $f'(x)$ pour $x > 0$ :
- Pour $0 < x < \frac{1}{\sqrt{3}}$, $f'(x) < 0$, donc $f$ est décroissante sur cet intervalle.
- Pour $x > \frac{1}{\sqrt{3}}$, $f'(x) > 0$, donc $f$ est croissante sur cet intervalle.

Calculons maintenant les limites de $f(x)$ aux bornes de l'intervalle $(0, \infty)$ :
\[
\lim_{x \to 0^+} f(x) = 0^3 - 0 = 0,
\]
\[
\lim_{x \to \infty} f(x) = \infty^3 - \infty = \infty.
\]

De plus, à $x = \frac{1}{\sqrt{3}}$, on a :
\[
f\left(\frac{1}{\sqrt{3}}\right) = \left(\frac{1}{\sqrt{3}}\right)^3 - \frac{1}{\sqrt{3}} = \frac{1}{3\sqrt{3}} - \frac{1}{\sqrt{3}} = -\frac{2}{3\sqrt{3}}.
\]

Ainsi, l'image de $(0, \infty)$ par $f$ est l'intervalle $(-\frac{2}{3\sqrt{3}}, \infty)$.


% Pour déterminer $f((0, +\infty))$, nous devons analyser le comportement de $f$ sur $(0, +\infty)$.

% La fonction $f(x) = x^3 - x$ est croissante pour $x > 1$ et décroissante pour $0 < x < 1$. En évaluant les limites, nous avons :
% \[
% \lim_{x \to 0^+} f(x) = 0
% \]
% \[
% \lim_{x \to +\infty} f(x) = +\infty
% \]
% En évaluant $f(1) = 0$, et comme $f$ est monotone croissante pour $x > 1$, on en déduit que :
% \[
% f((0, +\infty)) = (0, +\infty)
% \]

\section*{Résumé}

La fonction $f(x) = x^3 - x$ n'est pas injective mais elle est surjective. Nous avons également déterminé que :
\[
f^{-1}([-1, 1]) = [-1, 1] \quad \text{et} \quad f((0, +\infty)) = (0, +\infty).
\]

\subsection*{Explication des étapes :}
\begin{enumerate}
    \item Injectivité : On vérifie si deux valeurs de $x$ distinctes peuvent donner la même valeur pour $f(x)$. En résolvant l’équation $f(x_1) = f(x_2)$, on montre que $f$ n’est pas injective.
    \item Surjectivité : On vérifie que pour toute valeur réelle, il existe un $x$ tel que $f(x)$ atteigne cette valeur en analysant les limites à l’infini.
    % \item Préimage d’un intervalle : On résout les inégalités pour déterminer les $x$ tels que $f(x)$ appartienne à l’intervalle $[-1, 1]$.
    % \item Image d’un intervalle : On étudie le comportement de $f$ pour $x > 0$ pour déterminer l’image de cet intervalle.
\end{enumerate}

\section*{Exercice 1.6}
Soit $\mathbb{Z}$ l'ensemble des entiers. Définissons la fonction $f_1 : \mathbb{Z} \to \mathbb{Z}$ par $f_1(n) = 2n$. Il est clair que $f_1$ est injective car si $2n_1 = 2n_2$ pour certains $n_1, n_2 \in \mathbb{Z}$, alors en divisant par 2 des deux côtés, on obtient $n_1 = n_2$. Ainsi, $f_1$ est injective.

Cependant, $f_1$ n'est pas surjective car son image est constituée uniquement des entiers pairs. En effet, si nous supposons qu'il existe un $n \in \mathbb{Z}$ tel que $f(n) = 2k+1$ (un entier impair), alors $2n = 2k + 1$. Cela impliquerait $1 = 2(n - k)$, ce qui est une absurdité car $2$ ne divise pas $1$. Par conséquent, $f_1$ n'est pas surjective.

Considérons maintenant la fonction $f_3 : \mathbb{R} \to \mathbb{R}$ définie par $f(x) = x^2$. Comme $f(-1) = f(1) = 1$, on constate que $f_3$ n'est pas injective. De plus, $f_3$ n'est pas surjective car il n'existe aucun $x \in \mathbb{R}$ tel que $f(x) < 0$ — autrement dit, les valeurs négatives ne sont jamais atteintes par $f_3$.

Cependant, si nous modifions l'ensemble d'arrivée $\mathbb{R}$ en $[0, \infty)$, alors $f_4 : \mathbb{R} \to [0, +\infty)$ devient surjective. En effet, toute valeur dans $[0, +\infty)$ est atteinte par $f_4$, comme on peut le voir sur le graphe de $y = x^2$.

\section*{Exercice 1.7.3 et 1.7.4}
Déterminons si les fonctions suivantes sont injectives, surjectives ou bijectives :
\begin{enumerate}
    \item $f_3: \mathbb{R}^2 \to \mathbb{R}^2$ définie par $(x,y) \mapsto (x+y, x-y)$;
    \item $f_4: \mathbb{R} \setminus \{1\} \to \mathbb{R}$ définie par $x \mapsto \frac{x+1}{x-1}$.
\end{enumerate}
\section*{Solution}

\subsection*{1. Étude de la fonction $f_3 : \mathbb{R}^2 \to \mathbb{R}^2$}

La fonction $f_3$ est définie par $f_3(x, y) = (x+y, x-y)$. Nous devons vérifier l'injectivité, la surjectivité et la bijectivité.

\subsubsection*{Injectivité}
Soit $(x_1, y_1)$ et $(x_2, y_2)$ deux éléments de $\mathbb{R}^2$ tels que $f_3(x_1, y_1) = f_3(x_2, y_2)$, c'est-à-dire :
\[
(x_1 + y_1, x_1 - y_1) = (x_2 + y_2, x_2 - y_2).
\]
Cela nous donne deux équations :
\[
x_1 + y_1 = x_2 + y_2 \quad \text{et} \quad x_1 - y_1 = x_2 - y_2.
\]
En additionnant ces deux équations, on obtient $2x_1 = 2x_2$, donc $x_1 = x_2$. En soustrayant, on obtient $2y_1 = 2y_2$, donc $y_1 = y_2$. Par conséquent, $f_3$ est injective.

\subsubsection*{Surjectivité}
Prenons un couple arbitraire $(a, b) \in \mathbb{R}^2$. Nous devons trouver $(x, y) \in \mathbb{R}^2$ tel que :
\[
f_3(x, y) = (a, b) \quad \text{soit} \quad (x+y, x-y) = (a, b).
\]
Cela revient à résoudre le système suivant :
\[
x + y = a \quad \text{et} \quad x - y = b.
\]
En additionnant ces deux équations, on obtient $2x = a + b$, donc $x = \frac{a+b}{2}$. En soustrayant, on obtient $2y = a - b$, donc $y = \frac{a-b}{2}$. Ainsi, pour tout $(a, b) \in \mathbb{R}^2$, il existe un couple $(x, y) \in \mathbb{R}^2$ qui vérifie $f_3(x, y) = (a, b)$. Par conséquent, $f_3$ est surjective.

\subsubsection*{Bijectivité}
Puisque $f_3$ est à la fois injective et surjective, elle est bijective.

\subsection*{2. Étude de la fonction $f_4 : \mathbb{R} \setminus \{1\} \to \mathbb{R}$}

La fonction $f_4$ est définie par $f_4(x) = \frac{x+1}{x-1}$. Nous allons également vérifier l'injectivité, la surjectivité et la bijectivité.

\subsubsection*{Injectivité}
Soit $x_1, x_2 \in \mathbb{R} \setminus \{1\}$ tels que $f_4(x_1) = f_4(x_2)$, c'est-à-dire :
\[
\frac{x_1+1}{x_1-1} = \frac{x_2+1}{x_2-1}.
\]
En croisant les produits, on obtient :
\[
(x_1+1)(x_2-1) = (x_2+1)(x_1-1).
\]
Développons les deux côtés :
\[
x_1x_2 - x_1 + x_2 - 1 = x_1x_2 - x_2 + x_1 - 1.
\]
En simplifiant, on obtient $-x_1 + x_2 = -x_2 + x_1$, ce qui donne $2x_1 = 2x_2$, donc $x_1 = x_2$. Ainsi, $f_4$ est injective.

\subsubsection*{Surjectivité}
Pour vérifier la surjectivité, prenons $1\in \R$, et considérons l'équation suivante $f_4(x) = 1$, soit :
\[
\frac{x+1}{x-1} = 1.
\]
Cela implique que $x+1 = x-1$, soit $2=0$, ce qui est absurde. Ainsi, $f_4$ n'est pas surjective.

\subsubsection*{Bijectivité}
$f_4$ est injective mais pas surjective, donc $f_4$ n'est pas bijective.




\section*{Exercice 1.8}
Vous pouvez trouver la solution ici: \url{http://exo7.emath.fr/ficpdf/fic00003.pdf}

\section*{Exercice 1.9}

Soit $E$ un ensemble non vide. Si $A$ est une partie de $E$, on appelle \textbf{fonction caractéristique} de $A$, notée $\chi_A$, l’application définie de $E$ dans $\mathbb{R}$ par :

\[
\chi_A(x) = 
\begin{cases} 
1 & \text{si } x \in A, \\
0 & \text{si } x \notin A.
\end{cases}
\]

\begin{itemize}
    \item[(1)] (a) Si $f = \chi_A$ avec $A \subset E$, vérifier que $A = f^{-1}(\{1\})$.
    
    (b) Soit $f : E \to \mathbb{R}$. Démontrer qu’il existe $A \subset E$ vérifiant $f = \chi_A$ si et seulement si $f(E) \subset \{0,1\}$, c'est-à-dire que $f$ ne prend que des valeurs dans $\{0,1\}$.

    \item[(2)] Soient $A$ et $B$ deux parties de $E$, et notons $f = \chi_A$ et $g = \chi_B$. Montrer que les fonctions suivantes sont les fonctions caractéristiques d’ensembles que l’on déterminera :
    \[
    1 - f \quad ; \quad f \cdot g \quad ; \quad f + g - f \cdot g.
    \]
\end{itemize}

\section*{Solution}

\subsection*{(1) Vérification et démonstration}

\subsubsection*{(a) Vérifier que $A = f^{-1}(\{1\})$.}

La fonction caractéristique $\chi_A$ est définie par :
\[
\chi_A(x) = 
\begin{cases} 
1 & \text{si } x \in A, \\
0 & \text{si } x \notin A.
\end{cases}
\]

L'ensemble $f^{-1}(\{1\})$ est constitué des éléments de $E$ pour lesquels la fonction $f$ prend la valeur $1$. Puisque $f = \chi_A$, on a :
\[
f^{-1}(\{1\}) = \{x \in E \mid \chi_A(x) = 1\} = \{x \in E \mid x \in A\} = A.
\]

Donc, $A = f^{-1}(\{1\})$. \\

\textbf{Explication :} Vous pouvez expliquer aux étudiants que la fonction caractéristique $\chi_A$ "marque" les éléments appartenant à $A$ avec la valeur $1$, tandis que les autres éléments de $E$ sont associés à la valeur $0$. En définissant $f^{-1}(\{1\})$, on trouve bien que cet ensemble correspond à $A$.

\subsubsection*{(b) Montrer qu'il existe $A \subset E$ tel que $f = \chi_A$ si et seulement si $f(E) \subset \{0,1\}$.}

\textbf{Sens direct :} Supposons que $f = \chi_A$ pour un certain $A \subset E$. Par définition de la fonction caractéristique, on a :
\[
f(x) = \chi_A(x) =
\begin{cases} 
1 & \text{si } x \in A, \\
0 & \text{si } x \notin A.
\end{cases}
\]
Ainsi, $f$ ne prend que les valeurs $0$ ou $1$, ce qui implique que $f(E) \subset \{0,1\}$. \\

\textbf{Sens réciproque :} Supposons que $f : E \to \mathbb{R}$ et que $f(E) \subset \{0,1\}$, c'est-à-dire que $f(x) = 0$ ou $f(x) = 1$ pour tout $x \in E$. Définissons l'ensemble $A = f^{-1}(\{1\})$, c'est-à-dire :
\[
A = \{x \in E \mid f(x) = 1\}.
\]
Dans ce cas, $f$ coïncide avec la fonction caractéristique $\chi_A$, car :
\[
f(x) =
\begin{cases} 
1 & \text{si } x \in A, \\
0 & \text{si } x \notin A.
\end{cases}
\]
Donc $f = \chi_A$. \\

\textbf{Explication :} Pour cette démonstration, vous pouvez expliquer aux étudiants que si une fonction $f$ ne prend que les valeurs $0$ et $1$, alors elle peut être interprétée comme une fonction caractéristique d'un certain ensemble $A \subset E$.

\subsection*{(2) Fonction caractéristique de nouveaux ensembles}

\subsubsection*{(a) $1 - f$}

Nous avons $f = \chi_A$, donc :
\[
1 - f(x) = 
\begin{cases} 
1 - 1 = 0 & \text{si } x \in A, \\
1 - 0 = 1 & \text{si } x \notin A.
\end{cases}
\]
Cela correspond à la fonction caractéristique du complémentaire de $A$ dans $E$, noté $A^c$. Ainsi :
\[
1 - f = \chi_{A^c}.
\]

\subsubsection*{(b) $f \cdot g$}

La multiplication de $f$ et $g$ est définie par :
\[
f(x) \cdot g(x) = \chi_A(x) \cdot \chi_B(x).
\]
Cela vaut $1$ si $x \in A$ et $x \in B$, c'est-à-dire si $x \in A \cap B$. Sinon, cela vaut $0$. Donc :
\[
f \cdot g = \chi_{A \cap B}.
\]

\subsubsection*{(c) $f + g - f \cdot g$}

Calculons cette expression :
\[
f(x) + g(x) - f(x) \cdot g(x) = \chi_A(x) + \chi_B(x) - \chi_A(x) \cdot \chi_B(x).
\]
Si $x \in A \cup B$, alors le résultat vaut $1$. En effet :
- Si $x \in A$ mais $x \notin B$, le terme vaut $1 + 0 - 0 = 1$.
- Si $x \in B$ mais $x \notin A$, le terme vaut $0 + 1 - 0 = 1$.
- Si $x \in A \cap B$, le terme vaut $1 + 1 - 1 = 1$.
- Si $x \notin A \cup B$, le terme vaut $0$.

Ainsi, cette expression est la fonction caractéristique de $A \cup B$ :
\[
f + g - f \cdot g = \chi_{A \cup B}.
\]



\subsection*{Explication}
\begin{enumerate}
    \item Fonction caractéristique : Expliquer que la fonction caractéristique permet de représenter un ensemble en associant la valeur $1$ aux éléments appartenant à cet ensemble, et $0$ aux autres éléments.
	\item Vérification de  $A = f^{-1}(\{1\})$  : Montrez que l’ensemble des éléments qui sont envoyés vers 1 par  $\chi_A$  est précisément l’ensemble A, en raison de la définition même de la fonction caractéristique.
    \item Existence d’une fonction caractéristique : Expliquez que pour toute fonction qui ne prend que les valeurs $0$ et $1$, on peut associer un ensemble pour lequel cette fonction est la fonction caractéristique.
	\item Opérations sur les fonctions caractéristiques : Reliez les opérations sur les fonctions caractéristiques (soustraction, produit, somme) avec les opérations correspondantes sur les ensembles (complémentaire, intersection, union).
\end{enumerate}


\section*{Exercice 1.11}
En utilisant la formule du binôme de Newton, montrer que $$\sum_{k=0}^n (-1)^k \binom{n}{k} = 0.$$ En deduire la valuer de $$\sum_{k=0}^{2k} \binom{n}{2k}.$$
\section*{Solution}
Nous devons utiliser la \textbf{formule du binôme de Newton}, qui est donnée par :
\[
(x + y)^n = \sum_{k=0}^n \binom{n}{k} x^{n-k} y^k.
\]

En particulier, lorsque \(x = 1\) et \(y = -1\), nous avons :

\[
(1 + (-1))^n = \sum_{k=0}^n \binom{n}{k} 1^{n-k} (-1)^k.
\]

Or, \(1 + (-1) = 0\), donc :

\[
0^n = \sum_{k=0}^n \binom{n}{k} (-1)^k.
\]

\textbf{Conclusion :} Pour tout entier \(n \geq 1\), nous obtenons :

\[
\sum_{k=0}^n (-1)^k \binom{n}{k} = 0.
\]

\section*{Explication}

Cette équation est un résultat classique qui découle directement du développement binomial. La somme alternée des coefficients binomiaux avec les signes \((-1)^k\) est égale à 0, car l'expansion de \((1 + (-1))^n\) conduit à zéro pour tout \(n \geq 1\). Ce résultat est souvent appelé l'identité de la somme alternée des coefficients binomiaux.

\section*{Deuxième partie : Somme des coefficients binomiaux de rang pair}

On dois déduire la valeur de la somme :

\[
\sum_{0 \leq 2k \leq n} \binom{n}{2k}.
\]

Pour cela, considérons à nouveau la formule du binôme de Newton, mais cette fois en utilisant \(x = 1\) et \(y = 1\) :

\[
(1 + 1)^n = \sum_{k=0}^n \binom{n}{k} = 2^n.
\]

Cependant, cette somme peut être décomposée en deux parties. On peut écrire cette somme en deux parties, la première est lorsque $k$ est pair, et la deuxième partie est lorsque $k$ est impair: les termes où \(k\) est pair, et les termes où \(k\) est impair. Ainsi, on a :

\[
2^n = \sum_{0 \leq 2k \leq n} \binom{n}{2k} + \sum_{0 \leq 2k+1 \leq n} \binom{n}{2k+1}.
\]

Mais en utilisant la première identité démontrée, on sait que la somme des termes impairs est égale à la somme des termes pairs, car la somme alternée est nulle. Par conséquent, on peut écrire :

\[
\sum_{0 \leq 2k \leq n} \binom{n}{2k} = \frac{2^n}{2} = 2^{n-1}.
\]

\textbf{Conclusion :} La somme des coefficients binomiaux de rang pair est :

\[
\sum_{0 \leq 2k \leq n} \binom{n}{2k} = 2^{n-1}.
\]

\section*{Explication}

La deuxième partie de l'exercice repose sur l'idée que les coefficients binomiaux pairs et impairs sont distribués de manière symétrique. Cela nous permet de conclure que la somme des coefficients binomiaux de rang pair est la moitié de la somme totale des coefficients binomiaux, soit \(2^{n-1}\).

\subsection*{Explication des étapes }
    \begin{enumerate}
        \item Première partie :
	\begin{itemize}
	    \item On commence par appliquer la formule du binôme de Newton, $$((x + y)^n = \sum_{k=0}^n \binom{n}{k} x^{n-k} y^k),$$ qui permet de développer un binôme élevé à une puissance.
     \item En prenant $(x = 1)$ et $(y = -1)$, on simplifie la formule en obtenant une somme alternée de coefficients binomiaux.
     \item Cela nous conduit à la somme nulle car $((1 + (-1))^n = 0)$ pour tout $(n \geq 1).$
	\end{itemize}
     \item Deuxième partie :
     \begin{itemize}
		\item En utilisant la même formule du binôme, mais cette fois-ci avec $(x = 1)$ et $(y = 1)$, on développe $((1 + 1)^n = 2^n)$.
	      \item 	Cette somme totale peut être divisée en deux : les termes où l’indice $(k)$ est pair et ceux où $(k)$ est impair.
       \item Puisque la somme des coefficients binomiaux alternés est nulle, cela signifie que les sommes des termes pairs et impairs sont égales.
       \item On en déduit que la somme des termes pairs est égale à la moitié de la somme totale, soit $2^{n-1}$.
     \end{itemize}
     
    \end{enumerate}


\section*{Exercice 1.12}
Soit \( E \) un ensemble à \( n \) éléments et soit \( m \) un entier strictement positif. Déterminer :

\begin{enumerate}
    \item Le nombre d'éléments de \( E^m \).
    \item Le nombre de parties de \( E^m \).
\end{enumerate}

\section*{Solution}

\subsection*{1. Le nombre d'éléments de \( E^m \)}

L'ensemble \( E^m \) représente le \textbf{produit cartésien} de \( E \) par lui-même \( m \) fois. En d'autres termes, un élément de \( E^m \) est un \( m \)-uplet de la forme :

\[
(x_1, x_2, \dots, x_m) \quad \text{avec} \quad x_i \in E \ \text{pour} \ i = 1, 2, \dots, m.
\]

Comme \( E \) a exactement \( n \) éléments, pour chaque \( x_i \), il y a \( n \) choix possibles. Ainsi, le nombre total d'éléments dans \( E^m \), noté \( |E^m| \), est :

\[
|E^m| = n^m.
\]

\textbf{Conclusion :} Le nombre d'éléments de \( E^m \) est \( n^m \).

\subsection*{2. Le nombre de parties de \( E^m \)}

Le nombre de parties d'un ensemble \( A \) est donné par \( 2^{|A|} \), où \( |A| \) représente le nombre d'éléments de l'ensemble \( A \). Ici, l'ensemble \( E^m \) contient \( n^m \) éléments, comme démontré précédemment.

Par conséquent, le nombre de parties de \( E^m \), noté \( \mathcal{P}(E^m) \), est :

\[
\mathcal{P}(E^m) = 2^{|E^m|} = 2^{n^m}.
\]

\textbf{Conclusion :} Le nombre de parties de \( E^m \) est \( 2^{n^m} \).

\subsection*{Explication des étapes}

\begin{enumerate}
    \item \textbf{Première partie :}
    \begin{itemize}
        \item \( E^m \) est le produit cartésien de l’ensemble \( E \) par lui-même \( m \) fois. Cela signifie que chaque élément de \( E^m \) est une séquence de longueur \( m \), où chaque élément de la séquence provient de \( E \).
        \item Comme \( E \) a \( n \) éléments, pour chaque position dans la séquence, il y a \( n \) choix possibles. Le nombre total de séquences (ou d’éléments dans \( E^m \)) est donc \( n^m \).
    \end{itemize}
    
    \item \textbf{Deuxième partie :}
    \begin{itemize}
        \item Le nombre de parties d’un ensemble est déterminé par le nombre de sous-ensembles possibles. Pour un ensemble contenant \( k \) éléments, le nombre de sous-ensembles est \( 2^k \) (car pour chaque élément, il existe deux possibilités : l’inclure dans le sous-ensemble ou ne pas l’inclure).
        \item Dans notre cas, l’ensemble \( E^m \) contient \( n^m \) éléments, donc le nombre total de parties de cet ensemble est \( 2^{n^m} \).
    \end{itemize}
\end{enumerate}




\section*{Exercice 1.18}
En utilisant la fonction $x \to (1 + x)^n$, calculer :

\[
\sum_{k=0}^n k \binom{n}{k}, \quad \sum_{k=0}^n \frac{1}{k+1}\binom{n}{k}.
\]

\section*{Solution}

\subsection*{1. Calcul de $\sum_{k=0}^n k \binom{n}{k}$}

Le développement binomial de $(1 + x)^n$ est :

\[
(1 + x)^n = \sum_{k=0}^{n} \binom{n}{k} x^k.
\]

En dérivant cette expression par rapport à $x$ :

\[
\frac{d}{dx}\left( (1 + x)^n \right) = n(1 + x)^{n-1}
\]

et

\[
\frac{d}{dx}\left( \sum_{k=0}^{n} \binom{n}{k} x^k \right) = \sum_{k=1}^{n} k \binom{n}{k} x^{k-1}.
\]

En multipliant les deux membres par $x$, on obtient :

\[
n x (1 + x)^{n-1} = \sum_{k=1}^{n} k \binom{n}{k} x^k.
\]

Pour $x = 1$, nous avons :

\[
n(1 + 1)^{n-1} = \sum_{k=1}^{n} k \binom{n}{k} = n 2^{n-1}.
\]

Ainsi, la somme est :

\[
\sum_{k=0}^n k \binom{n}{k} = n 2^{n-1}.
\]

\subsection*{2. Calcul de $\sum_{k=0}^n \frac{1}{k+1} \binom{n}{k}$}

En partant de $(1 + x)^n = \sum_{k=0}^{n} \binom{n}{k} x^k$, on intègre cette expression :

\[
\int_0^x (1 + t)^n \, dt = \sum_{k=0}^{n} \binom{n}{k} \frac{x^{k+1}}{k+1}.
\]

L'intégrale donne :

\[
\int_0^x (1 + t)^n \, dt = \frac{(1 + x)^{n+1} - 1}{n+1}.
\]

En évaluant pour $x = 1$, nous obtenons :

\[
\frac{2^{n+1} - 1}{n+1} = \sum_{k=0}^{n} \frac{1}{k+1} \binom{n}{k}.
\]

Ainsi, la somme est :

\[
\sum_{k=0}^n \frac{1}{k+1} \binom{n}{k} = \frac{2^{n+1} - 1}{n+1}.
\]

\subsection*{Explication des étapes}
    \begin{enumerate}
        \item Première somme :	Nous avons utilisé la fonction génératrice $(1 + x)^n$ et dérivé l’expression pour obtenir une relation avec la somme $\sum_{k=0}^n k \binom{n}{k}$. Ensuite, nous avons substitué $x = 1$ pour obtenir le résultat final $n 2^{n-1}$.
        \item Deuxième somme :
		Nous avons intégré la fonction $(1 + x)^n$ et utilisé le fait que l’intégration produit une somme de la forme $\sum_{k=0}^n \frac{1}{k+1} \binom{n}{k}$. L’évaluation de l’intégrale à $x = 1$ nous donne le résultat $\frac{2^{n+1} - 1}{n+1}$.
    \end{enumerate}

















\end{document}
